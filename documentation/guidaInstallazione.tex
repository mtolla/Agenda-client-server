\documentclass[12pt]{scrartcl}
\usepackage{listings}
\usepackage{epigraph}

\title{Guida all'Installazione}
\subtitle{Ingegneria del Software}
\author{Stefano Ravetta, Simone Renzo, Matteo Scarpone, Mattia Tollari}
\date{29 Febbraio, 2016}
\begin{document}
\maketitle
\section{Informazioni su questo documento}
    Questo documento \`e la guida all'installazione del software sviluppato.
    Contiene le istruzioni da seguire per poter installare su varie piattaforme
    le dipendenze richieste.
\section{Ottenere i permessi necessari}
    \subsection{Gnu/Linux}
        \`E necessario dare i seguenti comandi da un emulatore di terminale.
        Nel caso sia attivo il login da utente root:
        \begin{verbatim}
        su -
        \end{verbatim}
        nel caso invece si abbiano i privilegi di sudouser:
        \begin{verbatim}
        sudo su -
        \end{verbatim}
    \subsection{Windows, Mac OS X}
        \`E necessario aver effettuato il login con utente che abbia i permessi
        di installare software.
\section{Installazione dipendenze}
\subsection{Installazione di Python}
    
    \subsubsection{Gnu/Linux}
    Nella maggior parte di distribuzioni l'interprete Python risulta preinstallato.
    Nel caso non si disponesse di un interprete Python \`e possibile installarlo dai
    gestori di pacchetti. \`E richiesto Python2.
    \begin{itemize}
        \item{Debian}
            \begin{verbatim}
apt-get install python2.7
            \end{verbatim}
    \item{Arch Linux}
            \begin{verbatim}
pacman -Sy python2
            \end{verbatim}
        \item{Gentoo}
            \begin{verbatim}
emerge --ask dev-lang/python:2.7
        \end{verbatim}
    \item{Ubuntu}
            \begin{verbatim}
apt-get install python2.7
            \end{verbatim}
    \item{Fedora}
            \begin{verbatim}
yum install python27
            \end{verbatim}
    \end{itemize}
    Nel caso fosse necessario installare Python compilando il sorgente, i sorgenti possono
    essere scaricati da: \\
    http://www.python.org/ftp/python/2.7.6/Python-2.7.6.tar.xz
    Seguono i comandi per la compilazione ed installazione
    \begin{verbatim}
tar xvzf Python-2.7.6.tar.xz
cd Python-2.7.6
./configure --prefix=~/python-2.7.6 --enable-shared
make
make install
    \end{verbatim}
    \subsubsection{Windows}
    Scaricare ed installare il pacchetto "python2.7" disponibile alla pagina:\\
    https://www.python.org/downloads/windows/
    \subsubsection{Mac OS X}
    Scaricare ed installare il pacchetto:\\
    https://www.python.org/ftp/python/2.7.11/python-2.7.11-macosx10.6.pkg\\
    Per Mac OS X a 64 bit
    o il pacchetto:\\
    https://www.python.org/ftp/python/2.7.11/python-2.7.11-macosx10.5.pkg\\
    per Mac OS X a 32 bit
    \subsection{Installazione di PyQt}
        \subsubsection{Gnu/Linux}
        Usando le moderne distribuzioni Linux \`e possibile instsallare PyQt usando
        il gestore di pacchetti della propria distribuzione. Segue un elenco di 
        comandi che richiedono al gestore di pacchetti di installare la dipendenza.
        I comandi seguenti devono essere dati con i privilegi di superuser:\\
        \begin{itemize}
            \item Debian
            \begin{verbatim}
apt-get install python-qt4
            \end{verbatim}
            \item Arch Linux
            \begin{verbatim}
pacman -Sy pyqt
            \end{verbatim}
            \item Gentoo
            \begin{verbatim}
emerge PyQt4
            \end{verbatim}
            \item Fedora
            \begin{verbatim}
apt-get install python-qt4
            \end{verbatim}
            \item Ubuntu
             \begin{verbatim}
yum install PyQt4
            \end{verbatim}
        \end{itemize}
        \`E possibile installare PyQt scaricando i sorgenti e compilandoli (sconsigliato).
        Scaricare i sorgenti da
        \\http://sourceforge.net/projects/pyqt/files/PyQt4/PyQt-4.11.4/PyQt-x11-gpl-4.11.4.tar.gz .
        \\Seguono comandi per generiche distribuzioni Linux.
        \begin{verbatim}
tar xvzf PyQt-x11-gpl-4.11.4.tar.gz
cd PyQt-x11-gpl-4.11.4
python2 configure-ng.py
make
make install
        \end{verbatim}
    \subsection{Flask}
    \subsubsection{Gnu/Linux}
        \begin{itemize}
            \item Debian
            \begin{verbatim}
apt-get install python-flask
            \end{verbatim}
            \item Arch Linux
            \begin{verbatim}
pacman -Sy python2-flask
            \end{verbatim}
            \item Gentoo
            \begin{verbatim}
emerge flask
            \end{verbatim}
            \item Ubuntu
            \begin{verbatim}
apt-get install python-flask
            \end{verbatim}
            \item Fedora
             \begin{verbatim}
yum install python-flask
            \end{verbatim}
        \end{itemize}
        
        \subsubsection{Pyton-pip}
        Nel caso non si disponesse di un gestore di pacchetti \`e consigliata l'installazione
        tramite python-pip.
        \begin{verbatim}
pip install flask
        \end{verbatim}
        \`E possibile usare pip anche tramite Windows o Mac OS X.
\end{document}

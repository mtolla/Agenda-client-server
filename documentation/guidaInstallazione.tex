\documentclass[12pt]{scrartcl}
\usepackage{listings}
\usepackage{epigraph}
\usepackage{graphicx}
\usepackage{booktabs}


\title{Guida all'Installazione}
\subtitle{Universit\`a degli Studi Milano Bicocca Ingegneria del Software}
\author{Stefano Ravetta, Simone Renzo, Matteo Scarpone, Mattia Tollari}
\date{29 Febbraio, 2016}
\begin{document}
\maketitle
\centerline{\includegraphics[scale=0.5]{ITicon.png}}
\tableofcontents
\section{Informazioni su questo documento}
    Questo documento \`e la guida all'installazione del software sviluppato.
    Contiene le istruzioni da seguire per poter installare su varie piattaforme
    le dipendenze richieste.

\section{Dipendenze Richieste}
  Elenco delle dipendenze:\\\\
  \begin{table}[h!]
  \label{tab:table1}
  \begin{tabular}{|l|l|l|}
      \toprule
      \textbf{Diependenza} & \textbf{Versione} &\textbf{Link}\\
      \midrule
      Python & 2.7.x & https://www.python.org/downloads/\\
      PyQt & 4.11.4 & https://sourceforge.net/projects/pyqt/\\
      fask & 0.10.1 & http://flask.pocoo.org/ \\
      requests & 2.9.1 & http://docs.python-requests.org/\\
      \midrule      
  \end{tabular}
  \end{table}
  \\\\\\La dipendenza PyQt pu\`o essere installata dal gestore di pacchetti delle varie distribuzioni
  Gnu/Linux e dall'installer che si trova sul sito ufficiale (Windows). Su  Windows, durante l'installazione di Python,
  sar\`a necessario spuntare la voce che chiede se esportare il Path.
  Le altre dipendenze possono essere installate tramite python-pip\footnote{https://pypi.python.org/pypi/pip}
  tramite la sintassi 
  \begin{verbatim}
  pip install <package name>
  \end{verbatim}
  
        \`E possibile usare pip anche tramite Windows o Mac OS X.
\section{Scaricamento ed uso del software}
Clonare tramite git il repository.
\begin{verbatim}
git@github.com:UnimibSoftEngCourse1516/progetto-gruppo4.git
\end{verbatim}
Eseguire il file che avvier\`a il server:
\begin{verbatim}
python2 progetto-gruppo4/startServer.py 
\end{verbatim}
Ora \`e possibile (da un'altra shell)
    \begin{itemize}
        \item Avviare l'applicazione per la creazione dei nuovi utenti:
        \begin{verbatim}
python progetto-gruppo4/administration.py
        \end{verbatim}
        \item Avviare il Client
        \begin{verbatim}
python progetto-gruppo4/startClient.py
        \end{verbatim}
    \end{itemize}
Segue un elenco di utenti gi\`a creati con i propri ruoli.
      \begin{table}[h!]
  \label{tab:table1}
  \begin{tabular}{|l|l|l|}
      \toprule
      \textbf{Utente} & \textbf{Password} &\textbf{Ruolo}\\
      \midrule
      carluca & carluca & project-manager\\
      stefano & stefano & teamleader \\
      mattia & tolla & partecipante\\
      \midrule      
  \end{tabular}
  \end{table}
    \\

\subsection{Analisi con sonarqube}
Segue la tabella che contiene gli strumenti per usare sonarqube e la
risorsa dal quale scaricarli.\\
\\
\begin{table}[h!]
  \label{tab:table1}
  \begin{tabular}{|l|l|}
      \toprule
      \textbf{Componente} & \textbf{Link}\\
      \midrule
      python plugin & http://docs.sonarqube.org/display/PLUG/Python+Plugin\\
      sonar scanner & http://docs.sonarqube.org/display/SCAN/Analyzing+with+SonarQube+Scanner\\
      \midrule      
  \end{tabular}
  \end{table}\\
Nel repository \`e presente il file che contiene la configurazione per avviare correttamente sonarscanner.
\end{document}


\documentclass[12pt]{scrartcl}
\usepackage{booktabs}

\title{Casi d'Uso}
\subtitle{Ingegneria del Software}
\author{Stefano Ravetta, Simone Renzo, Matteo Scarpone, Mattia Tollari}
\date{03 Marzo, 2016}
\begin{document}
\maketitle

\begin{table}[h!]
  
  \caption{AddHolidays}
  \label{tab:table1}
  \begin{tabular}{|l|l|}
    \toprule
    \textbf{Sezione del caso d'Uso} & \textbf{Commento}\\
    \midrule
    \textbf{Nome del caso d'Uso} & AddHolidays\\
    \midrule
    \textbf{Portata} & Sistema di registrazione utente\\
    \midrule
    \textbf{Livello} & Livello utente\\
    \midrule
    \textbf{Attore primario} & Qualsiasi tipo di utente \\& (Developer, TeamLeader, ProjectManager).\\
    \midrule
    \textbf{Parti interessate e Interessi} & L'utente \`e interessato ad 
    \\ & inserire i propri giorni di vacanza.\\
    \midrule
    \textbf{Pre-condizioni} & L'utente deve essere presente nella base di dati \\
    \midrule
    \textbf{Garanzia di successo} & L'utente deve essere connesso al \\ & server ed aver effettuato il login. \\
    \midrule
    \textbf{Scenario principale di successo} & L'utente, dopo aver effettuato il login sul server,
    \\& inserisce i propri periodi di  vacanza che vengono \\& registrati nella base di dati. \\
    \midrule
    \textbf{Estensione} & Pu\`o fallire se l'utente inserisce \\& una data di inzio maggiore di una data di fine.\\
    \midrule
    \textbf{Requisiti speciali} & - \\
    \midrule
    \textbf{Elenco delle varianti} & L'utente inserisce le varie date di inizio e fine 
    \\& in un'interfaccia grafica Qt, ma \`e \\& possibile usare qualsiasi altra interfaccia.\\ 
    \textbf{tecnologiche e dei dati} & \\
    \midrule
    \textbf{Frequenza di ripetizione} & Poco frequente in quanto questo caso d'uso \\& verr\`a chiamato solo alla registrazione di un utente.\\
    \bottomrule
  \end{tabular}
\end{table}

\begin{table}[h!]
    \caption{ViewDevelopersActivities}
  \label{tab:table2}
  \begin{tabular}{|l|l|}
    \toprule
    \textbf{Sezione del caso d'Uso} & \textbf{Commento}\\
    \midrule
    \textbf{Nome del caso d'Uso} & ViewDevelopersActivities\\
    \midrule
    \textbf{Portata} & Sistema di consultazione \\
    \midrule
    \textbf{Livello} & Livello utente\\
    \midrule
    \textbf{Attore primario} & Qualsiasi tipo di utente \\& (Developer, TeamLeader, ProjectManager).\\
    \midrule
    \textbf{Parti interessate e Interessi} & L'utente \`e interessato a \\& consultare l'agenda.\\
    \midrule
    \textbf{Pre-condizioni} & L'utente deve essere connesso al server.\\
    \midrule
    \textbf{Garanzia di successo} & L'utente deve aver effettuato il login.\\
    \midrule
    \textbf{Scenario principale di successo} & L'utente, dopo aver effettuato il login sul server,
    \\& invia la richiesta per visualizzare l'agenda ed 
    \\& il server invia i dati richiesti.\\
    \midrule
    \textbf{Estensione} & - \\
    \midrule
    \textbf{Requisiti speciali} & - \\
    \midrule
    \textbf{Elenco delle varianti}\\ \textbf{tecnologiche e dei dati} & Le richieste sono indipendenti dall'interfaccia usata.\\
    \midrule
    \textbf{Frequenza di ripetizione} & Molto frequente in quanto l'operazione pu\`o \\& 
    essere richiamata pi\`u volte da diversi utenti in tempi brevi.\\
    \midrule
    \textbf{Varie} & - \\
    \bottomrule
  \end{tabular}
\end{table}

\begin{table}[h!]
  \caption{MenageSubgroups}
  \label{tab:table3}
  \begin{tabular}{|l|l|}
    \toprule
    \textbf{Sezione del caso d'Uso} & \textbf{Commento}\\
    \midrule
    \textbf{Nome del caso d'Uso} & MenageSubgroups\\
    \midrule
    \textbf{Portata} & Sistema di amministrazione utenti\\
    \midrule
    \textbf{Livello} & Livello utente\\
    \midrule
    \textbf{Attore primario} & TeamLeader\\
    \midrule
    \textbf{Parti interessate e Interessi} & Il TeamLeader vuole creare, eliminare \\& o modificare un gruppo di lavoro.\\
    \midrule
    \textbf{Pre-condizioni} & L'utente dev'essere online e \\& deve aver effettuato l'accesso al server.\\
    \midrule
    \textbf{Garanzia di successo} & Le operazioni sui gruppi devono essere 
    \\& coerenti con la base di dati:\\& 
    non \`e possibile modificare o eliminare un gruppo 
    \\& che non esiste.\\
    \midrule
    \textbf{Scenario principale di successo} & L'utente, dopo aver effettuato il login sul server,
    \\& invia la richiesta per gestire i gruppi; 
    \\& verra mostrata un'interfaccia periodi
    \\& a seconda dell'operazione scelta. 
    \\& Verr\`a inviata la richiesta al server che
    \\& provveder\`a a modificare la base di dati opportunamente.\\
    \midrule
    \textbf{Estensione} & - \\
    \midrule
    \textbf{Requisiti speciali} & - \\
    \midrule
    \textbf{Elenco delle varianti}\\ \textbf{tecnologiche e dei dati} & I metodi non dipendono dall'interfaccia grafica usata.\\
    \midrule
    \textbf{Frequenza di ripetizione} & Frequenza media.\\
    \midrule
    \textbf{Varie} & - \\
    \bottomrule
  \end{tabular}
\end{table}


\begin{table}[h!]
  \caption{ManageIndividualActivites}
  \label{tab:table4}
  \begin{tabular}{|l|l|}
    \toprule
    \textbf{Sezione del caso d'Uso} & \textbf{Commento}\\
    \midrule
    \textbf{Nome del caso d'Uso} & MenageIndividualActivities\\
    \midrule
    \textbf{Portata} & Sisema amministrazione attivit\`a\\
    \midrule
    \textbf{Livello} & Livello utente\\
    \midrule
    \textbf{Attore primario} & TeamLeader, Developer\\
    \midrule
    \textbf{Parti interessate e Interessi} & L'utente \`e interessato a modificare il piano attivit\`a.\\
    \midrule
    \textbf{Pre-condizioni} & L'utente deve essere online ed aver effettuato il login\\
    \midrule
    \textbf{Garanzia di successo} & Le operazioni devono essere 
    \\& coerenti con la base di dati.\\
    \midrule
    \textbf{Scenario principale di successo} & L'utente, dopo aver effettuato il login sul server,
    \\& invia la richiesta per gestire le attivit\`a individuali;
    \\& a seconda dell'operazione richeista 
    \\& (aggiungere, modificare od eliminare un'attivit\`a)
    \\& verr\`a presentata un'interfaccia che recuperer\`a ed
    \\& invier\`a i dati al server che li memorizzer\`a nella base di dati.\\
    \midrule
    \textbf{Estensione} & \\
    \midrule
    \textbf{Requisiti speciali} & -\\
    \midrule
    \textbf{Elenco delle varianti}\\ \textbf{tecnologiche e dei dati} & La gestione di flusso di rete e dati \`e
    \\& indipendente dall'interfaccia utente usata.\\
    \midrule
    \textbf{Frequenza di ripetizione} & Frequenza medio-alta.\\
    \textbf{Varie} & -\\
    \bottomrule
  \end{tabular}
\end{table}


\begin{table}[h!]
  \caption{ManageProjects}
  \label{tab:table5}
  \begin{tabular}{|l|l|}
    \toprule
    \textbf{Sezione del caso d'Uso} & \textbf{Commento}\\
    \midrule
    \textbf{Nome del caso d'Uso} & MenageProjects\\
    \midrule
    \textbf{Portata} & Sistema di amministrazione progetti\\
    \midrule
    \textbf{Livello} & Livello utente\\
    \midrule
    \textbf{Attore primario} & ProjectManager\\
    \midrule
    \textbf{Parti interessate e Interessi} & Il ProjectManager \`e interessato ad aggiungere,
    \\& modificare od eliminare un progetto.\\
    \midrule
    \textbf{Pre-condizioni} & L'utente deve essere connesso, aver effettuato il login
    \\& ed avere i permessi necessari.\\
    \midrule
    \textbf{Garanzia di successo} & L'utente deve inserire operazioni coerenti con la base di dati.\\
    \midrule
    \textbf{Scenario principale di successo} & L'utente, dopo aver effettuato l'accesso al server,
    \\& sceglie l'attivit\`a da eseguire. L'interfaccia mostrata
    \\& ricever\`a i dati e li trasmetter\`a al server che li 
    \\& inserir\`a nella base di dati.\\
    \midrule
    \textbf{Estensione} & - \\
    \midrule
    \textbf{Requisiti speciali} & - \\
    \midrule
    \textbf{Elenco delle varianti}\\ \textbf{tecnologiche e dei dati} & I metodi non dipendono dall'interfaccia.\\
    \midrule
    \textbf{Frequenza di ripetizione} & Frequenza medio-bassa.\\
    \textbf{Varie} & - \\
    \bottomrule
  \end{tabular}
\end{table}

\begin{table}[h!]
  \caption{ManageGroupActivities}
  \label{tab:table6}
  \begin{tabular}{|l|l|}
    \toprule
    \textbf{Sezione del caso d'Uso} & \textbf{Commento}\\
    \midrule
    \textbf{Nome del caso d'Uso} & ManageGroupActivities\\
    \midrule
    \textbf{Portata} & Sistma di amministrazione attivit\`a \\
    \midrule
    \textbf{Livello} & Livello utente\\
    \midrule
    \textbf{Attore primario} & TeamLeader, ProjectManager. \\
    \midrule
    \textbf{Parti interessate e Interessi} & Il ProjectManager o TeamLeader \`e interessato a
    \\& gestire (modificare, aggiungere od eliminare) attivit\`a di gruppo.\\
    \midrule
    \textbf{Pre-condizioni} & L'utente deve essere connesso al server ed
    \\& aver effettuato il login\\
    \midrule
    \textbf{Garanzia di successo} & L'utente deve avere i permessi necessari 
    \\& per svolgere l'operazione e deve effettuare richieste
    \\& coerenti con la base di dati.\\
    \midrule
    \textbf{Scenario principale di successo} & L'utente accede al server, compie le azioni desiderate
    \\& tramite la GUI, il server riceve i dati e li scrive nella base di dati.\\
    \midrule
    \textbf{Estensione} & - \\
    \midrule
    \textbf{Requisiti speciali} & - \\
    \midrule
    \textbf{Elenco delle varianti}\\ \textbf{tecnologiche e dei dati} & Il software non dipende dall'interfaccia.\\
    \midrule
    \textbf{Frequenza di ripetizione} & Frequenza medio-bassa\\
    \textbf{Varie} & - \\
    \bottomrule
  \end{tabular}
\end{table}


\begin{table}[h!]
  \caption{ManageWorkgroups}
  \label{tab:table7}
  \begin{tabular}{|l|l|}
    \toprule
    \textbf{Sezione del caso d'Uso} & \textbf{Commento}\\
    \midrule
    \textbf{Nome del caso d'Uso} & ManageWorkgroups\\
    \midrule
    \textbf{Portata} & \\
    \midrule
    \textbf{Livello} & \\
    \midrule
    \textbf{Attore primario} & \\
    \midrule
    \textbf{Parti interessate e Interessi} & \\
    \midrule
    \textbf{Pre-condizioni} & \\
    \midrule
    \textbf{Garanzia di successo} & \\
    \midrule
    \textbf{Scenario principale di successo} & \\
    \midrule
    \textbf{Estensione} & \\
    \midrule
    \textbf{Requisiti speciali} & \\
    \midrule
    \textbf{Elenco delle varianti}\\ \textbf{tecnologiche e dei dati} & \\
    \midrule
    \textbf{Frequenza di ripetizione} & \\
    \textbf{Varie} & \\
    \bottomrule
  \end{tabular}
\end{table}
\end{document}

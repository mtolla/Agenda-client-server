\documentclass[12pt]{scrartcl}
\usepackage{listings}
\usepackage{epigraph}

\title{Scelte Implementative}
\subtitle{Ingegneria del Software}
\author{Stefano Ravetta, Simone Renzo, Matteo Scarpone, Mattia Tollari}
\date{29 Febbraio, 2016}

\begin{document}
\maketitle


\section{Introduzione}	% Produces section heading.  Lower-level sections
			% are begun with similar \subsection and
			% \subsubsection commands; numbering is automatic!

\subsection{Informazioni su questo ducumento}
In questo documento verranno trattate e discusse le scelte
implementative che sono state prese per la realizzazione del progetto "Team Diary Management"
e le motivazioni delle stesse .

\subsection{Contenuti}
Si porr\`a l'attenzione in modo particolare su
\begin{itemize}
    \item Software per la creazione dei diagrammi
    \item Software per la stesura della documentazione
    \item Linguaggio di programmazione
    \item Formato e struttura della base di dati
    \item Librerie per la creazione dell'interfaccia utente
    \item Struttura dei sistemi di sicurezza adottati
\end{itemize}

\section{Linguaggio}
\subsection{Introduzione}
\'E stato scelto il linguaggio di programmazione Python per alcune caratteristiche
piuttosto interessanti che lo possono far risaltare nell'ecosistema dei linguaggi di programmazione odierno.
\subsection{Alto livello}
Python consente di programmare ad un livello molto alto: con poche righe riesce a gestire
in modo semplice e diretto operazioni complesse senza che il codice risulti troppo complesso da leggere e comprendere.
\\ \\
Questo ha permesso al team di sviluppo di concentrarsi di pi\`u sulle parti pi\`u astratte
e progettuali (come i design pattern e la rimozione di vari code smell) che
su questioni pi\`u vicine alla comprensione del funzionamento del linguaggio.
\subsection{Filosofia}
Una parte di un famoso "easter egg" inserito nell'interprete Python (che pu\`o apparire appena si
prova ad importare la libreria "this") recita:
\begin{quotation}
If the implementation is hard to explain, it's a bad idea.
\end{quotation}

\begin{quotation}
If the implementation is easy to explain, it may be a good idea.
\end{quotation}
Queste righe sembrano denotare in modo sarcastico la filosofia del linguaggio.
Uno degli obiettivi del linguaggio \`e appunto far s\`i che il codice scritto
abbia il miglior compromesso tra stringatezza e comprensibilit\`a. Si intende
quindi che codice estremamente compatto non \`e sinonimo di efficienza a lungo termine
in quanto provocher\`a quasi sicuramente un dilatamento delle tempistiche
sia fasi di sviluppo successive sia in fase di testing ed ottimizzazione.
Un esmpio che pu\`o far notare il modo in cui questo tipo di filosofia inserito
nella struttura del linguaggio possa effettivamente influire sul codice \`e 
quello dell'indentazione: per forzare il programmatore ad indentare sempr
e il codice (e a farlo in modo corretto) questo linguaggio non usa caratteri
di begin ed end nella definzione di classi, funzioni e procedure,
espressioni condizionali, espressioni iterative (come fa ad esempio Java con le
parentesi graffe) ma usa solo l'indentazione. In tal modo da una parte i programmatori
che di norma non indentano il loro codice risulteranno "costretti" a farlo,
e dall'altra i programmatori che gia` di norma indentano il proprio codice
non dovranno inserire altri simboli per esprimere ci\`o che concettualmente
pu\`o essere gi\`a presente nell'indentazione. \\ \\
Secondo il nostro team di sviluppo questa caratteristica risultava molto
interessante per la progettazione di un progetto in cui e\` richiesta
particolare attenzione alla qualit\`a del codice prodotta. Inoltre a livello
pratico tutto ci\`o e\` risultato addirittura comodo dato che ogni membro
del gruppo aveva un proprio stile di stesura del codice e si e\` notata la
necessit\`a di uno stile univoco con cui lavorare.

\begin{thebibliography}{10}	


\end{thebibliography}
\end{document}             % End of documen

\documentclass[12pt]{scrartcl}
\usepackage{listings}
\usepackage{epigraph}
\usepackage{dirtree}

\title{Scelte Implementative}
\subtitle{Ingegneria del Software}
\author{Stefano Ravetta, Simone Renzo, Matteo Scarpone, Mattia Tollari}
\date{29 Febbraio, 2016}

\begin{document}
\maketitle


\section{Introduzione}	% Produces section heading.  Lower-level sections
			% are begun with similar \subsection and
			% \subsubsection commands; numbering is automatic!

\subsection{Informazioni su questo ducumento}
In questo documento verranno trattate e discusse le scelte
implementative che sono state prese per la realizzazione del progetto "Team Diary Management"
e le motivazioni delle stesse .

\subsection{Contenuti}
Si porr\`a l'attenzione in modo particolare su
\begin{itemize}
    \item Software per la creazione dei diagrammi
    \item Software per la stesura della documentazione
    \item Linguaggio di programmazione
    \item Formato e struttura della base di dati
    \item Librerie per la creazione dell'interfaccia utente
    \item Struttura dei sistemi di sicurezza adottati
\end{itemize}

\section{Linguaggio di Programmazione}
\subsection{Introduzione}
\'E stato scelto il linguaggio di programmazione Python\footnote{https://www.python.org/} per alcune caratteristiche
piuttosto interessanti che lo possono far risaltare nell'ecosistema dei linguaggi di programmazione odierno.
\subsection{Alto livello}
Python consente di programmare ad un livello molto alto: con poche righe riesce a gestire
in modo semplice e diretto operazioni complesse senza che il codice risulti troppo complesso da leggere e comprendere.
\\ \\
Questo ha permesso al team di sviluppo di concentrarsi maggiormente sulle parti pi\`u astratte
e progettuali (come i design pattern e la rimozione di vari code smell) rispetto
a questioni pi\`u vicine alla comprensione del funzionamento del linguaggio.
\subsection{Filosofia}
Una parte di un famoso "easter egg"\footnote{Questo easter egg, "The Zen of Python"
\`e stato inserito nell'interprete Python e che pu\`o apparire appena si prova ad 
importare la libreria "this".
}
recita:
\begin{quotation}
If the implementation is hard to explain, it's a bad idea.
\end{quotation}

\begin{quotation}
If the implementation is easy to explain, it may be a good idea.
\end{quotation}
Queste righe sembrano denotare con toni sarcastici la filosofia del linguaggio.
Uno degli obiettivi del linguaggio \`e appunto far s\`i che il codice scritto
risulti il miglior compromesso tra stringatezza e comprensibilit\`a. Si intende,
quindi, che codice estremamente compatto non \`e sinonimo di efficienza a lungo termine
in quanto provocher\`a quasi sicuramente un dilatamento delle tempistiche
sia in fasi di sviluppo successive, sia in fase di testing ed ottimizzazione.
Un esempio che pu\`o far notare il modo in cui questo tipo di filosofia inserito
nella struttura del linguaggio possa effettivamente influire sul codice \`e 
quello dell'indentazione: per forzare il programmatore ad indentare sempre
il codice (e a farlo in modo corretto) questo linguaggio non usa caratteri
di begin ed end nella definzione di classi, espressioni condizionali,
espressioni iterative, funzioni e procedure (come fa ad esempio Java con le
parentesi graffe) ma usa solo l'indentazione. In tal modo, da una parte i programmatori
che di norma non indentano il loro codice risulteranno "costretti" a farlo,
e dall'altra i programmatori che gia` di norma indentano il proprio codice
non dovranno inserire altri simboli per esprimere ci\`o che concettualmente
pu\`o essere gi\`a espresso dall'indentazione. \\ \\
Secondo il nostro team di sviluppo, questa caratteristica risulta molto
interessante per la gestione di un progetto in cui \`e richiesta
particolare attenzione alla qualit\`a del codice prodotto. Inoltre, a livello
pratico, tutto ci\`o \`e risultato addirittura comodo perche` si \`e mostrata la
necessit\`a di lavorare con uno stile di stesura del codice unico (ogni membro
del gruppo ne presentava uno diverso).

\subsection{Portabilit\`a}
Python \`e un linguaggio precompilato e il suo interprete \`e disponibile per
numerose\footnote{L'elenco delle piattaforme supportate \`e reperibile a https://www.python.org/download/other/} piattaforme. 
La presenza di questa caratteristica non \`e stata sicuramente
ignorata dato che quasi tutti i membri del team di sviluppo usano piattaforme ed ambienti
diversi. Anche la distribuzione dell'applicazione e di pacchetti contenenti le 
varie dipendenze risulta estremamente semplificata grazie alla portabilit\`a di Python.


\subsection{Dizionari}
Una delle funzionalit\`a che sono state pi\`u apprezzate per la stesura del codice
relativa al progetto \`e senza dubbio la gestione dei dizionari che Python offre
e la loro semplice e diretta interazione con classi e anche dati Json.


\section{Librerie per la creazione dell'interfaccia utente}
\subsection{Introduzione}
Sono state scelte le librerie Qt\footnote{http://www.qt.io/} che fanno parte di uno storico
framework open source. 
\subsection{Potenzialit\`a}
Le librerie Qt sono in grado di gestire interfacce grafiche, basi di dati, 
attivit\`a multimediali, flussi di rete, gestione di file e notifiche di sistema.
Una caratteristica piuttosto interessante di queste librerie \`e che possono essere usate
attraverso molti\footnote{\`E possibile visionare tutti i linguaggi supportati a questa pagina: 
http://wiki.qt.io/Category:LanguageBindings}
linguaggi di programmazione e non sono legate a nessuna piattaforma specifica.
Permettono anche lo sviluppo di applicazioni mobile\footnote{
il progetto SailfishOs pu\`o dare un'idea delle potenzialit\`a delle librerie Qt:
http://sailfishos.org/develop/sdk-overview/}
\subsection{Ambienti di sviluppo}
Per lo sviluppo dell'interfaccia grafica \`e stato comodo usare lo strumento
QtDesigner\footnote{http://doc.qt.io/qt-4.8/designer-manual.html} (che segue il paradigma WYSIWYG) 
che permette di disegnare GUI, modificare le propriet\`a dei vari
oggetti e generare codice XML dell'interfaccia realizzata.
\subsection{interazione con Python}
Per far interagire il file XML generato da QtDesigner con Python
\`e stato usato lo strumento Pyuic4\footnote{http://pyqt.sourceforge.net/Docs/PyQt4/designer.html}
che svolge il lavoro meccanico di conversione da file che descrive un'interfaccia
ad uno script Python che mette a disposizione classi che rappresentano
gli oggetti usati ed i metodi  per accedervi.

\section{Base di dati}
\subsection{Introduzione}
    Dato che la base di dati usata ha una struttura molto semplice \`e stato scelto
    di non usare uno strumento complesso come un database relazionale. 
    \`E stata creata una base di dati in formato Json che \`e risultata semplice da
    strutturare e da gestire. \`E risultato comodo interagire assieme sulla base
    di dati tramite Git. L'unico problema riscontrato \`e stato dover gestire i vari
    file dal codice dell'applicazione. Questo problema \`e stato praticamente risolto
    con l'introduzione di una struttura di metodi e costanti.
    Con un database relazionale sarebbe stato necessario usare
    i file ottenuti dal dump del database ed importarli in locale ad ogni modifica.
    L'ideazione e la gestione del database relazionale avrebbero occupato pi\'u tempo
    della soluzione adottata.
\subsection{Struttura}
    La struttura della base di dati \`e cos\`i definita:
    
    \dirtree{%
    .1 database.
    .2 user.json.
    .2 location.json.
    .2 activity.json.
    .2 group.json.
    .2 project.json.
    }
    
    \subsubsection{User.json}
        Contiene i dati relativi all'utente: id, username, password, nome, cognome, gruppi (una lista che
        contiene l'id del gruppo e il ruolo ("partecipante" o "team leader"), vacanze (una listas che
        contiene un id, un nome, data di inizio e data di fine).
    
    \subsubsection{Location.json}
        Contiene i dati relativi ai luoghi: un id ed un nome.
    
    \subsubsection{Activity.json}
        Contiene i dati relativi alle attivit\`a: id, nome, id del progetto di riferimento, 
        id dell'utente che ha creato l'attivit\`a, tipo di attivit\`a (se di gruppo, di progetto
        o singola), l'id del luogo, id del grupppo, lista di id dei partecipanti.
    
    \subsubsection{Group.json}
        Contiene i dati relativi ai grupi: id, nome, father (False se il gruppo non \`e un sottogruppo,
        id del sottogruppo altrimenti), lista degli id dei sottogruppi.
        
\subsection{Interazione con Python}
    Per importare e gestire la base di dati dal programma sono stati usati i dizionari costruiti,
    come Json, con [chiave: valore]. Questo tipo di dato ha reso molto agile
    l'accesso e la generazione di codice Json.


\section{Struttura dei sistemi di sicurezza adottati}
\subsection{Codifica delle password}
    Per evitare di salvare le password degli utenti in chiaro
    \`e stato fatto ricorso ad un algoritmo di hashing one-way:
    esiste una funzione semplice da calcolare che associa ad ogni stringa un hash
    ma non esiste una funzione semplice da calcolare che associa ad ogni hash una stringa
    tale che, se applicata la funzione iniziale, l'hash ottenuto sia quello di partenza.
    \`E stato scelto SHA512 in quanto oggi risulta quello pi\`u robusto a 
    livello di problemi di collisione e preimmagine. Per l'implementazione in Python
    \`e stata usata la libreria Hashlib.
\subsection{Token, autenticazione e gestione delle sessioni}
    Per gestire autenticazione e sessioni \`e stato implementato un sistema di
    token. Un token \`e un hash SHA512 di utente, password e timestamp. Con un controllo
    su questi tre campi si impedisce ad un utente malintenzionato (che non dispone dei una
    password valida) di poter accedere ad una sessione gi\`a attiva. 
    Il timestamp serve per imporre scadenze al token (limite attualmente posto
    uguale ad un giorno). Cos\`i facendo risulter\`a difficile anche provare un
    attacco che mira a scoprire il token dato che in poco tempo i tentativi precedenti
    risulteranno inutili.
\subsection{Password dimenticate}
    Nel caso venisse smarrita la password di un utente verr\`a visualizzato
    un messaggio che indicher\`a di contattare gli amministratori di 
    sistema. \`E stata presa questa scelta perch\`e non erano presenti
    le risorse necessarie per implementare e testare a fondo un sistema
    di recupero password che usi email o SMS.

\end{document}
